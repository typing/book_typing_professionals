\onecolumn

\part*{あとがき}
\addcontentsline{toc}{section}{あとがき}

表紙を見てティンと来た方も多いと思いますが、
本書はコンピュータ業界では知らぬものがいないと言っても良い知名度と信頼度を誇る、
某出版社の書籍を参考にして作っています。
コンピュータ業界に比べれば非常に小さな世界ではありますが、
タイピング界において本書が多くの人に役立つことを祈って。\\

さて、夏コミが終わった直後、主催者のノリだけで始まった本書の企画ですがいかがだったでしょうか。タイピングに関して一家言を持っている執筆陣が集結したとはいえ、やはりそれぞれ独自研究の域を脱していないのは事実です。そんな荒削りな記事ばかりですが、あなたの琴線に触れるような記事が一つでもあればこれ以上の喜びはありません。

一方で、この記事で書いてることは本当に正しいのか?そんな疑問を持つことも当然あるでしょう。そんなときはtwitterなりブログなりでどんどんツッコミを入れてください。荒削りの記事はあなたとタイピング界のさらなる議論によって磨かれ、より多くの人にとって価値のある知識へと昇華していく可能性を秘めています。さらにはその過程であなた自身の中に潜む原石を見つけられるかもしれません。本書をきっかけにしてより深い議論を重ねて、タイピングの技術や知識、そして楽しさを多くの人で共有していけることを願っています。

最後になりましたが、突然の執筆の提案に率先して参加していただいた執筆者のみなさんには本当に感謝しています。タイピング界のメンツが集まればあんなことやこんなことが書けるだろうなと妄想していたことが、ここまで現実にできるとは本当に思ってもいませんでした。本書の製作を通じて一つでもみなさんの経験につながれば幸いです。

\begin{flushright}
2011年12月 tomoemon
\end{flushright}

