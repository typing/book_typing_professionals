\articlepart{英語タイピング}{テル(vuttar)}

\section{はじめに}

日本国内での「タイピング」と言えば、圧倒的な主流は日本語タイピング(日本語の文章もしくは単語をタイピングする)です。普段の生活で日本語を用いている我々日本人としては、タイピング競技においても日本語入力を基本とするのは当然でしょう。しかし英語タイピングには英語タイピングの魅力が沢山あり、日本人であっても挑戦する価値は十分にあると思います。

この記事は、主に
\begin{enumerate}
 \item 日本国内の英語タイピング競技紹介(2章)
 \item 海外の英語タイピング競技紹介(3章)
 \item 英語タイピングの魅力(4章)
 \item 英語タイピングの特徴や上達のコツ(5章)
\end{enumerate}
これらの内容をまとめたものです。皆さんが英語タイピングに興味を持ち、また実際に英語タイピングを始めてみるきっかけとなれば幸いです。

\section{日本国内での英語タイピング}

日本国内でも英語タイピングのランキングはいくつか運営されていて、それなりの盛り上がりを見せています。主なものを5つ紹介し、競技としての特性をそれぞれ簡単にまとめます。

\subsection{美佳テキスト}

10年以上前からタイピング練習ソフトとして人気を博している「美佳タイプ」の姉妹ソフト。幸福の王子、不思議の国のアリスなど、6種類の名作英文(の抜粋)をタイピングします。ランキングでは、それぞれのテキストの打鍵速度と、全テキストの平均打鍵速度を競います。ここ数年は新規参加者・記録更新者がかなり少なくなっているように感じますが、課題文や競技特性のオーソドックスさは大きな魅力で、未プレイであれば是非試してみてほしいソフトです。また、英語タイピングを始めたばかりの方には、腕試しとしてまず美佳テキストをプレイしてみることをお勧めします。

競技特性:修正あり(ミスタイプをしたら \key{BackSpace} で修正)、ミス減点なし(ミスしたら修正しなくては進めない)、完全固定文(6つのテキストから選択して打つ)、長文、記号あり(, . : など)、更新可(記録を何度も塗り替えられる)

\subsection{タイプウェル英単語}

言わずと知れた人気タイピング練習ソフト「タイプウェル」シリーズのひとつ。詳細な打鍵ログの表示・詳細な練習記録の保存・分析補助機能など、他の追随を許さない充実した機能、かゆいところに手の届くストレスフリーな練習環境、参加者が多く項目も多岐に渡る本格的なランキングが魅力です。タイプウェル英単語には基本英単語1500・拡張A-F・拡張G-P・拡張Q-Zの4モードがあり、各モードごとに出題単語が決まっていて、その中からランダムで選択された単語を400文字分打っていきます。単語の間ではスペース入力を要するため、“英単語”と言っても、打鍵感的には“文章”入力との違いは少ないです。

競技特性:修正なし、ミス減点なし、ランダム単語、400字固定、記号ほぼなし(' のみ)、更新可

\subsection{タイプウェル憲法 英語モード}

同じくタイプウェルシリーズのひとつで、日本国憲法の英訳をタイピングします。1章分を打ち終わるまで休みなしの「通し」モード、1条項ずつ選んで打つ「条項別」モードがあります。タイプウェルらしい詳細な記録と、ちまちま得点を伸ばしていける着実さが魅力です。ランキングでは、通しの記録と条項別の記録(通しの途中で条項別の最高記録を出した場合も、得点に反映されます)を総合した得点を競います。参加者はタイプウェル英単語に比べると少なめですが、超長文入力の練習・腕試しの手段として非常に優れていると思います。

競技特性:修正なし、ミス減点なし、完全固定文、長文~短文、記号あり、更新可

\subsection{e-typing 英語}

人気タイピング練習サイト「e-typing」内の英語タイピング練習モード。e-typing トップページ→タイピング*バラエティ→英語→腕試しレベルチェック からプレイ可能です。おおむね4~10単語程度の短文が順次出題され、ミス減点を加味したスコアで競います。ランキング参加者が多く、手軽に腕試ししたい場合に非常に向いています。また、英語タイピングでは珍しい短文競技なので、短文の練習がしたい方、もしくは初期加速を鍛えたい方には向いていると思います。

競技特性:修正なし、ミス減点あり、ランダム文、短文、記号あり、更新可

\subsection{毎日パソコン入力コンクール 英文A・英文B}

毎日パソコン入力コンクール(以下毎パソ)は、毎日新聞社が主催するパソコン入力競技の大会です。大手新聞社が主催し、総務省や文部科学省、全国の教育委員会などが後援に名を連ねています。少なくとも日本国内で一般的に参加できるものの中では、唯一の“公的権威”の裏付けを持った競技だと言えると思います。また中学生であれば英文A、高校生であれば英文Bの全国大会(予選大会上位者のみで行われる決勝大会)への参加権があり、中学生・高校生の方にとっては非常に挑戦しがいのある大会だと思います。“パソコン入力”コンクールというだけあって、純粋なタイピング能力ではなく(比較的)実用を重視した入力速度を競います。具体的には「ミスをしてもその場では教えてくれない」「ミスをしても先に進める」「文字飛ばしや行飛ばしをしていても先へ進める」などの特徴があり、攻略には独特の対策が必要になります。

毎パソの英語入力部門である「英文A」「英文B」では、それぞれ入試問題からの抜粋・毎日新聞社説の英訳が事前に配布され、課題文として出題されます。公式の練習ソフトと課題文を用いて本番まで練習し、本番期間中に5分間の一発勝負で記録を出し、その入力文字数を競います。誤字・脱字があると大きく減点される(更に、ミスが一定数を超えると失格となる)ため、速度だけではなく正確に入力する能力(またはミスタイプに気付いて修正する能力)が重要となります。ランキングに参加するためには大会参加料が必要ですが、課題文のダウンロード・練習ソフトの使用は無料のため、英文入力能力の腕試しや練習に利用するだけであればお金は必要ありません。

競技特性:完全実用入力、ミス減点あり、固定文、長文、記号あり、一発勝負

\subsection{総括}

「競技特性」の部分にそれぞれのソフトの特徴をまとめてきましたが、以上5つの日本国内の競技だけでも、かなり競技としてのバリエーションに富んでいることが分かると思います。これらを組み合わせるだけでバランス良く練習することができますし、自分の好みに合う競技を選択することもできます。日本語タイピングに比べれば競技人口は今ひとつ少ないですが、競技としての下地はそれなりに整っているので、国内の競技だけでも十分楽しめるとは思います。

\section{海外での英語タイピング}

海外では(当たり前ですが)英語タイピングが主流です。競技人口は相当多く、レベルの高いタイパーもわんさといて、非常に魅力的な環境だと思います。私もそれほど海外のタイピング競技に詳しいわけではないですが、日本でも有名になっているサイトがいくつかあるので、それを紹介します。

\subsection{Typeracer}

他の競技とは一線を画す、対戦型のタイピング競技サイトです。タイピングをレースに模して、課題文を打ち切る早さを競います。課題文は候補の中から毎回ランダムに出題され、それを打ち切る早さで順位が決まります。勝敗の他に毎回のレースで出した速度も記録され、最近10レースの平均速度と、過去全レースの平均速度でランキングを競います。その他に「最近1時間以内に出された記録」のランキングもあり、比較的簡単にトップページに載って目立つことができます。

700 種類を超える課題文の内容は曲の歌詞・エッセイ・何やらカタい文章など多岐に渡っていて、かなり充実しています。課題文の数が少ないとテキスト慣れの要素が大きくなりすぎるため、これは大きな魅力だと思います。実力の近いタイパーと自動で試合を組んでくれる上、Sean Wrona 氏\footnote{Typeracer の頂点に立つタイパー。Typing Zone でもトップランカー。米国のタイピング大会での優勝経験もあり、世界トップのタイパーと言っても過言ではない。}をはじめとする超上級タイパーから初心者まで、様々な実力のタイパーが多数プレイしているため、初心者から上級者まで、気軽に緊迫したレースを楽しめます。

Typeracer での打鍵速度の基準となる単位は WPM ( Words Per Minute : 1分間に何単語打ったか)です。英語タイピング界で一般的に使われている単位ですが、これは非常に曲者で、実は「実際に打った単語数」とは全く関係がありません。スペースを含めた5打鍵を1単語と換算して、擬似的に「1分間に何単語打ったか」を示しているだけで、実質的には CPM ( Characters Per Minute : 1分間に何文字打ったか) を5で割ったものに他なりません。ともかく、Typeracer での WPM に5を掛ければ、CPM に変換することができます。

競技としてもゲームとしても非常に優秀なサイトですが、「レースを途中退席するとその回の記録が無効になる」という残念な特徴があります。これを利用して実力以上に記録を高めることもできるため、ランキングの信頼性は若干低くなってしまっています。ただ、それでも基本的に上位陣の実力は本物だと感じますし、実際にマッチングすれば、相手の記録が本物かどうかは手応えで分かります。この問題を踏まえても十分に面白く、やりがいもあるため、個人的には断然一押しのタイピング競技サイトです。

競技特性:修正あり、ミス減点なし、固定文からランダム出題、中長文、記号あり、更新可

\subsubsection*{注釈: CPM と KPM}

CPM という単位がいきなり出てきましたが、この単位に馴染みのない方も多いかと思います。CPM は、海外の英語タイピングで主に使われる入力速度の単位です。対して日本での英語タイピング競技には KPM ( Keystrokes Per Minute : 1分間に何“キー”打ったか)もしくは SPM ( Strokes Per Minute : KPM と同じ)が使われています。恐らく日本語タイピングで KPM ( SPM )が使われる影響でしょう。

ところで、日本での英語タイピング競技における KPM ( SPM ) は、海外での CPM と同じ意味で、つまり「1分間に打った文字数」をカウントする単位として使われています。この用法は正しいのでしょうか。英語には\key{Shift}入力があるため、打った文字数とキー数は一致しません(例えば "It's me." は 8文字で10キー)。しかし日本における KPM ( SPM )の考え方では、"It's me." を 8文字とみなしてしまいます。これは用法として妥当でないと私は考えます。

\subsection{Typing Zone - General ranking}

Typing Zone はフランス発のタイピング競技サイトで、フランス・アメリカを中心に様々な国のタイパーが参加しています。主要なランキングとしては General ranking と Championship の二つがあり、General ranking は記録が継続して残る通常のランキングです。現在の General ranking における上位ランカーの国籍を1位から見ていくと、ブラジル、ルーマニア、アメリカ、アメリカ、フランス、オーストラリア、セルビア、カナダ……となっていて、そのグローバル感は個人的には大きな魅力です。ランキングは登録しやすく(一度登録作業を済ませれば記録更新は1ボタン)、見やすく、登録者数もそれなりにいて、満足できる水準です。

General ranking では、フランス語文・英語文・円周率・アルファベット( A to Z )・アルファベット( Z to A )の5部門の総合得点を競います(部門別のランキングも用意されています)。A to Z と Z to A で総合ランキングの得点のうち合計30\%、円周率だけで20\%を占めるのは正直どうなんだ、という気はするのですが、ともかくこういうことになってしまっているので、まあそれはそれとして考えればなかなか面白いです。全部門、課題文は完全固定で、何度も更新できるようになっています。「ということは単なるテキスト慣れゲーじゃないか」と言われるとその通りなのですが、後述の Championship があるので、「General ranking はそういうものなんだ」と開き直れば、やはり結構楽しめるのです。

Typeracer と同じく Typing Zone でも打鍵速度の計測単位には WPM が使われていますが、Typing Zone ではスペースを含めた”6打鍵”を1単語として計算しているようです。これについて公式に言及している文章は見つかりませんでしたが、他の日本人タイパーがそう言っていたこと、また私が独自に(多少おざなりなやり方ですが……)検証してみたことから、ほぼ間違い無いかと思います。非常に分かりにくい仕様なので、Typeracer と統一してほしいです。

競技特性:修正あり、ミス減点なし、完全固定文、中長文、記号あり(,.のみ)、更新可

\subsection{Typing Zone - Championship}

Championship は毎年開催される大掛かりなタイピング競技ツアーのようなもので、独特の大会形式を採用しています。まず"Master"と呼ばれる大会が毎月(1年で 12回)開催され、Master毎にランキングが公開されます。各 Master の上から半分くらいまでには成績に従って100~1点のポイントが割り振られ、Championship では1年間の合計得点を競います。

各 Master の課題文は一般ユーザーの提案・投票で選ばれます。各月毎に課題文の長さ・言語に指定があるため、課題文のバリエーションはある程度保証されています。(1月:短文・言語自由、2月:中文・フランス語、3月:長文・英語 など。)また6,10月は英語・フランス語以外の文章が選ばれることになっているため、ある程度の公平性も確保されています。

課題文が決定されてから記録送信が締め切られるまでの時間は1~2週間程度なので、延々とやり込んでワード慣れで記録を伸ばすということも難しく、かと言ってランダム文のような運の要素も少ないので、競技としてかなりバランスが取れています。1年がかりの大会ということでランキングの推移を見るのも楽しく、エンタテインメント性に優れたランキングです。ただし、ある程度上位でないと大会ポイントがつかない今の仕様は改善の余地があるとは思います。ともあれ、個人的にはこの Championship のような大会がタイピング競技の理想形なんじゃないかな、と考えていたりします。

競技特性:修正あり、ミス減点なし、固定文、中~中長文、記号あり、更新可

\subsection{Typingweb(タイピング練習サイト)}

タイピング練習ソフトにはしばしば\texttt{fff jjj ddd kkk sss lll aaa ;;;}、\texttt{fff jjj fff jjj ffj jjf fjf jfj}のような基本パターンを延々と打って練習するような初心者向けモードがありますが、Typingweb はそういったトレーニングを手軽に&段階的にできるようにデザインされた、いわゆる"Typing Tutor"系の練習サイトです。こういった基礎トレーニングの効果を疑問に感じる人も多いかと思うのですが、私は比較的効果があると考えています。特に英語タイピングではこういった基礎トレーニングが効果的だと思っています。理由としては以下を挙げておきます。
\begin{itemize}
 \item 打鍵パターンの組み合わせが多様(ローマ字入力では基本的に子音→母音→子音→母音と打っていくのに対して、英語ではそうはいかない)なため、様々な打鍵パターンに対応できなければならない(→基礎トレーニングをすると苦手な打鍵パターンがよく分かる)
 \item 知らない(ほとんどランダム入力に思えるような)単語が出てくることも多いため、アルファベット単位で認識してゴリ押しするような能力もときに重要になる(この能力は基礎トレーニングである程度鍛えられる)
\end{itemize}

Typingweb は「上段キーの練習」「下段キーの練習」といったオーソドックスな練習から、記号の練習、大文字含みの単語の練習、頻出単語の練習、数字の練習、苦手な文字を含む単語の重点練習など、非常にバリエーション豊富な練習コースを有しています。必要なコースだけ練習するといったことももちろん可能なので、苦手分野を克服するため、もしくは苦手分野を発見する(これも非常に重要で、難しいことです)ために、ぜひ積極的に使ってみてほしいと思います。また、一応ランキングのようなものもあるので、競技として楽しむこともできるかもしれません。

\subsection{総括}

Typeracer の神ゲーぶりが際立ちます。ユーザー登録(無料)しなくてもゲストとして対戦できるので、是非気軽にやってみてください。対戦と言っても「対戦部屋に入ってチャットで挨拶して……」といった面倒なことは必要無い(トップページの"Enter a typing race"を押せば即レースが始まる)ので、英語が分からなくても安心です。ユーザ層は幅広いので初心者の方でも楽しめます。Typing Zone (の特に Championship )はある程度上達してからプレイした方が面白いかもしれません。

\section{英語タイピングの魅力}

ここまで英語タイピングの練習ソフトやランキングサイトを紹介してきましたが、現在日本語タイピングを楽しんでいる人にとっては、「結局英語タイピングの何がどう楽しいの?」というのが一番気になる部分ではないかと思います。好みは様々なので参考にならないかもしれませんが、ともかく私が感じている「英語タイピングの魅力」を紹介してみようと思います。

\subsection{世界中の人と競いあえる}

「練習の成果を競いあう」というのは、様々な競技に共通する楽しみ方です。タイピングでもそれは同じで、ランキングや対戦サイトを通じてタイピングの能力を競うことは、多くの人にとって主要な楽しみだと思います。競う楽しみの大小は、競う相手(の集団)の質と量で決まると思います。質というのはレベルの高さではなくて、多様性だと思っています。レベルが高い人から低い人、老人から子ども、正確性タイパーから乱打タイパー、男性に女性など、色々な人と色々な観点から競えること、これが魅力的な競技となる条件ではないでしょうか。

この考え方からいくと、様々な国の人間とタイピングを競えるというのは大きな魅力です。日本語タイピングに競技者が何人いても、「結局全員日本人なんだよな」と。いや日本人だけで競って何が悪い、といえばそうなんですが(競技人口もまあ致命的なほど少なくはないし)、やはりもっと色んな人と同じ土俵で競いたい、と思ってしまうんです。また、母語も文化も生育環境も全く違う人と同じ競技を楽しんでいる、ということに凄くロマンを感じます。

\subsection{競技の統一性}

タイピング競技は元々タイプライター・キーボードによる文章入力の能力を競うことから派生したものです。しかし文章入力の能力を競うと言っても、課題文はどうするか、ミス修正の有無・方法は、採点基準は、などと、競技に落とし込む過程で様々な差異が生まれます。こういった差異があると、それぞれの競技形式へと競技者が分散し、結果として同じ土俵で競う人数が減少します。異なる形式の支持者間での無用な争いも起こりえます。

日本語入力・日本語タイピングの場合、漢字変換の存在がこの差異を顕著にしています。漢字を変換するかしないか、するならば変換辞書はどうするか、また異なる漢字混じり文の間での記録比較はできるのか、などの問題があります。また、助詞などを用いて文章を間断なくつなげる言語特性から、単語単位のタイピングと文章単位のタイピングの間で大きな違いが発生します。例えば単語単位のタイピング競技は文章入力能力の測定としては意味を成さないのではないか、という問題です。そもそも文章入力能力の測定が目的とも限らないのですが、これとの親和性が低いと競技の求心力は下がります。また、文章単位の競技も、課題文慣れの要素が大きくなる、課題文選択に対する出題者の恣意性が大きくなるなど、やはり入力能力の測定手段としては妥当性に欠けています(まあ単語入力よりはマシだと思いますが)。このように一長一短で、これらの要素が日本語タイピング競技の差異を助長しています。

対して英語入力では、変換を必要としないため、変換の問題はクリアされています。大文字入力(シフトワーク)のあり・なしは統一性を損なっていますが、多くの競技では大文字の入力が要求されているため、影響は軽微です。単語と文章の問題は英語タイピングでも共通ですが、通常の文章にもスペースが入って単語が独立しているため、単語単位の競技でも、入力能力測定の手段としての妥当性は比較的高いでしょう。また文章単位でも、スペースの緩衝材があるため課題文慣れの要素は小さいと考えられます(特に初見文で負担が増す課題文認識・打鍵組立における猶予となるため。この考え方について詳しくは別記事:タイピング練習論で展開)。このようにどちらの形式で測られるタイピング速度も、一般的な「入力能力」とあまりかけ離れておらず、単語単位・文章単位の競技間の差異は実質的に小さくなっています。これらの影響で英語タイピングでは競技の統一性が高く、多くの人が同じ土俵で戦えます。この点はとても大きな魅力だと感じています。

ところで、これまで競技の差異を悪者のように言ってきましたが、この差異は言い換えれば多様性です。日本語入力には色々な競技があって面白い、と考えることはできますし、私も実際その魅力を感じてはいます。ただ、現在の競技人口を考えると、競技者が分散してしまうデメリットの方が大きいような気がするのです。

\section{英語タイピング攻略}

これまでの章では英語タイピング布教のようなことをやってきましたが、この章では主に英語タイピングを上達させるための Tips を挙げていきます。まだまだ私自身英語タイピングに関する経験と考察が不足していて、具体的な練習方法というよりは「何に気を付けるべきか」といった曖昧な示唆にとどまっていますが、ある程度の参考にはなるかと思っています。

\subsection{英語タイピングの特徴}

練習を始めるにあたって、まず英語タイピングの大まかな特徴を知っておくことが大事です。

第一の特徴として、左手の負担が大きいことが挙げられます。タイプウェル英単語 基本1500や拡張A-Fなどで簡単に実感できますが、\key{C}, \key{D}, \key{E}, \key{S} などを筆頭に左手担当キーが多く、左手キーが連続する打鍵パターンも多いため、左手の動作能力や持久力が重要となります。普段から左手の無駄なく速い動きを意識しましょう。打ち方次第だとは思いますが、個人的には、左手は少し力を抜いて軽快に打つ(打鍵は軽く、離鍵をしっかり)ことがコツかなと思います。

第二の特徴として、打鍵パターンの多様性が挙げられます。ローマ字入力のように 子音→母音→子音→母音 という決まったパターンが無いので、多様な打鍵パターンが現れ混乱させられます。

そこで多様な打鍵パターンに対応する能力が必要になりますが、それには正確で素早い先読みが大前提となります。「どうもペースを維持して打てない打鍵パターンが多い(すぐ詰まる)」と思ったら、あえて速度を落として先読みに猶予を与えることで、多様なパターンに対応できて全体としては逆に速度が上がる、ということもあり得ます。またリプレイの確認や基礎練習を通して苦手なパターンを探ったり、苦手なパターンを反復練習することも重要になるでしょう。

\subsection{英語力との関連は?}

「英語ができないと英語タイピングでは不利なのかな」と不安に思う方は多いようです。私の意見としては、英語が苦手な場合、
\begin{itemize}
 \item (特に練習開始期の)文字認識の速度・正確性において若干の不利がある
 \item 動作の速度や正確性においては不利は無い
 \item 英語に頻出の打鍵パターンを把握しづらいため、それらに対応する能力の上達において不利があるが、長期的には克服できる(注1)
 \item 初見長文では、先の展開を予測できず若干不利(と言っても日本語ですら意味を理解しづらい速度なので、影響は小さい)
\end{itemize}

総括としては、
\begin{itemize}
 \item しっかり練習していけば、長期的には大した不利は無い(注2)
 \item 「読めない文を打つのが辛い」という場合、モチベーションは保ちにくいかもしれない
\end{itemize}

この程度の影響はあるものの、基本的に「英語が分からなくても尻込みする必要はない」と言えると思います。

\subsubsection*{(注1)}
英単語の構成パターン(-ion とか -ant とか dec- とか、色々あります)を見慣れていないため、文字がバラバラに見えてしまい、パターン認識能力が上達しにくい、ということです。ただ、これは打っていく内に少しずつ分かってきて、ちゃんとつながりとして認識してスムーズに打てるようになるはずです。また第4節でこういったパターンの例をいくつか紹介します。実際に練習しながら、代表的なパターンを頭に染み込ませていけば、全く問題は無いと思います。

\subsubsection*{(注2)}
しっかり練習すれば不利を克服していけると言っても、成長速度には若干の差が出るでしょう。しかし練習の量や質、そして継続の重要性に比べれば、ほとんど無視できる差だと思います。

\subsection{英語タイピングの急所}

第1節で英語タイピングの特徴を2つ挙げましたが、英語タイピング上達のために気を付けるべきことはまだまだ沢山あります。この節ではそれらを一気に挙げていきます。どれも特効薬的な対処方法はありませんが、常に意識の端に置いておき、足を引っ張っているのは何かと自問し続けることで、上達のスピードは上がることと思います。

\subsubsection*{同キー連打}

英語タイピングでは同キーの2連打が頻出です。例えばタイプウェル英単語 基本1500では、出題元となる1500単語中234単語に同キー連打が含まれています。同キー連打は恐らくどうあがいても減速パターンとなってしまいますが、遅くなっても仕方ないと諦めて漫然と打つのと、少しでもタイムロスを短縮するために努力するのでは、タイムに大きな違いが出てくると思います。

同キー連打は、先読みで見つけた段階であらかじめ意識を割いておき、1打目の反動を使ってリズム良く2打目を打つことで、かなり改善できます。1打目を少し強めに打つのがコツかと思います。

かなり上達してきた段階でも、先読みが不十分な状態で同キー連打に突入すると、大幅に減速したり2打目が入らなかったりしがちです。普段から気を付けておき、先読み+連打の準備を自然にできるよう心がけていきましょう。

\subsubsection*{a, is, to などの短い単語}

英語には a, is, to, on, at, I, he など、1~2打分しかない単語がいくつかあります。これらの単語が出てくると、スペースが間に合わなくなって減速したり、タイミングがずれて詰まりがちです。もし特に苦手だと感じたら、短い単語が含まれる文章をタイプウェルFTなどで重点的に練習しましょう。これらは慣れれば大体加速ワードになるため、やりがいはあると思います。短い単語を早く打てるようになると、今度は後述の「ひっくり返りミス」が出やすくなってくるので、そちらにも注意が必要です。

\subsubsection*{シフトワーク}

英語タイピングでは、文頭・固有名詞の頭・強調される単語に使われる大文字、記号(" ' ?)など、\key{Shift} との同時押しを要求される場合が多くあります。\key{Shift} 入力は減速やミスにつながりやすく、少しでも素早く、無駄なく行うことが重要です。私もシフトワークは苦手なのですが、こればかりはとにかく練習あるのみだと思います。タイプウェルオリジナル 大小文字混在・すべてのキーなども利用しつつ、地道に練習していきましょう。また、打ちづらいパターンに \key{Shift} 入力が混じって出てきた場合や、リズムが狂っている状態で \key{Shift} 入力が出てきた場合など、ミスを避けるために思い切って減速する覚悟も時には必要になると思います。

\subsubsection*{修正}

タイプウェル、e-typing 以外の多くの英語タイピング競技では、ミスをした場合 \key{BackSpace} を使っての修正が必要になります。ミス修正自体のタイムロスに加え、また加速しなおすまでのタイムロスも生じるため、全体として非常に大きなロスになります。

修正によるロスを減らすためには、正確なタイピング能力、素早く修正する能力などを身につけることが基本ですが、ミスをしそうな時に思い切って減速する勇気も大事だと思います。また、自分はどの程度の速度で打てばどの程度のミスが出るのか、そしてどの程度の速度と正確性で打てば一番効率が良いのか、といったことをしっかり把握しておくことも非常に重要です。ミスを恐れず攻めにいくのか、ミスをしないように守っていくのか、といったその場その場の判断も重要になってくるでしょう。

\subsubsection*{\key{C}, \key{D}, \key{E} 付近の難パターン}

第1節で「左手の負担が重い」と書きましたが、難しい左手動作の中でも特に中指担当の \key{C} \key{D} \key{E} が絡んだ運指が難しく、かなりのタイムロスになってきます。decorate, education など、指を動かすスペースに余裕の無い中指が激しい上下動を強いられ、かなり詰まりやすいのです。標準運指の場合、どうあがいてもこれらは減速パターンになると思いますが、練習で改善できる部分も多いはずです。これらのパターンを含む単語を繰り返し練習するなどして、中指の動作を上達させていくことが必要だと思います。また、上への移動(\key{C}\key{D}, \key{D}\key{E}, \key{C}\key{E})よりも下への移動(\key{E}\key{D}, \key{D}\key{C}, \key{E}\key{C})の方が簡単なので、まず下への移動の高速化を意識してみるといいかもしれません。上への移動の場合は、指の先がキーに引っかかるミスが起きやすいです。爪を切る、指を立てすぎないなど、色々と気をつけてみてください。

\subsubsection*{ひっくり返りミスと打鍵チャンク}

例えば season → seaosn のように、前後の2打がひっくり返ってしまうミスのことを、ここでは「ひっくり返りミス」と呼びます。日本語タイピングでもしばしば起きるミスですが、英語タイピングでは特に頻発します。ひっくり返りミスは、全く違うキーを打ってしまう(かすってしまう)ミスと違って、打った瞬間に「ミスをした」と気付きにくく、大幅な詰まりの原因となります。個人的には、「チャンク認識の改善」によってこのミスを減らすことができると考えています。その考えを説明します。

まず、こういったミスが起きる原因について、先の season → seaosn を例にして考えます。まず \key{S}\key{E}\key{A}\key{S} は、私の運指では \finger{2312} となりますが、これは中々打ちにくいパターンです。\key{S}\key{E} の内(キーボードの内側の方)への流れから \key{A} への外への動きへの移行がまず難しく、次の \key{S} を \finger{2} でスムーズに押すことはかなり難しいと思います。
       
ここで少し話が飛びますが、高速で打鍵するためには、「キーを押した」という認識をしてから次のキーを押しているようでは間に合いません。「このぐらいのペースでキーを押せるだろう」という感覚に従って、「前のキーを押した」という確証が無いままに次のキーを押していくのです。そのため、難しい打鍵パターンが現れると、「もう押せているだろう」と思って次のキーを押したはいいが、実はまだ前のキーを打ち終わっていない、ということが起きます。つまり、難しい \key{S}\key{E}\key{A}\key{S} に手間取っている間に、今か今かと待ち構えていた \key{O} をつい押してしまい、結果として \key{S} と \key{O} がひっくり返ってしまう、ということなのです。

こういったミスは、打鍵チャンクが途切れた部分でよく起きます。打鍵チャンクというのは私が勝手に作った言葉ですが、「円滑な打鍵のため、打鍵列の一部をひとつのまとまりとして認識しなおしたもの」のことです(認識段階のチャンクとは別のものとなりますが、多分脳内では曖昧に運用されてると思うのであまり気にしなくてもいいと思います)。

分かりにくいと思うので例を出して説明します。高速タイピングでは、複数のキーを素早く正確に押下するために、いくつかのキーを一まとまりで認識して、そのまとまりを一息に打ち切る、ということが一般的に行われます。例えば condition という単語を、\key{C}-\key{O}-\key{N}-\key{D}-\key{I}-\key{T}-\key{I}-\key{O}- \key{N} という9つの独立したキーとして捉えるのではなく、\key{C}\key{O}\key{N}-\key{D}\key{I}-\key{T} \key{I}\key{O}\key{N} の3つ(3チャンク)に分けて考え、\key{C}\key{O}\key{N} をほぼ一息に、\key{D}\key{I} もほぼ一息に、\key{T}\key{I}\key{O}\key{N} もほぼ一息に打ち切る、といったやり方です。ピンとこなければ、\key{C}, \key{O}, \key{N} にそれぞれ指を乗せて、たたたっ とほぼ同時に(少しずつずらして)押してみてください(運指が違って上手くいかない人は別のパターンで試してみてください)。一連の \key{C}\key{O}\key{N} の打鍵がひとつの動作としてまとめられる、ということが分かると思います。この場合 \key{C}\key{O}\key{N}, \key{D}\key{I}, \key{T}\key{I}\key{O}\key{N} の3つが打鍵チャンクです。

打鍵チャンクは結局自分が打鍵パターンをどう認識するかなので、ある程度好きなように変えることができます。例えば cond-ition と認識してもいいですし、con-dition でも、co-ndit-ion でも構わないわけです。ただ、やはり認識の仕方によって打ちやすさは変わってきますし、おかしな打鍵チャンク認識は、先述のひっくり返りミスを誘発します。

先の season の例では、左手の \key{S} と右手の \key{O}\key{N} が別々のチャンクに入ってしまっていたことと、seas という難しい(難しすぎる)部分をチャンクにしてしまったことが問題です。seas に手間取っている内に、次のチャンクを打つために構えていた右手を動かしてしまったのです。seas という1チャンクの難易度が高すぎて、ペース感覚が乱されたとも言えます。例えば sea-son と認識しておけば、son が一まとまりなので、\key{S} と \key{O} がひっくり返ることはほぼあり得なかったでしょう。\key{A} と \key{S} がひっくり返ることは無いのかと言えば、これは両方とも左手なので大丈夫だと思います。(少し変則的ですが、se-ason と認識すれば更に良いかと思います。打ちにくい sea を更に分解して、右向きの流れを持つチャンクのみで打鍵を構成するやり方です。分離した \key{E}-\key{A} は打ちにくく若干減速しますが、個人的には ason の加速で十分補えると思います。)

このように、優れた(自分の運指に合った)打鍵チャンクの組み方を模索することで、ひっくり返りミスはかなり減らせると思います。また、単純な打鍵速度にも良い影響を及ぼすはずです。実際にはタイピングをしながら打鍵チャンクの組み方を考えるのは難しいので、普段の練習の際に少しずつ打ちやすい打鍵チャンクの組み方を考えていくことになります。詰まった単語や打鍵パターンをできるだけ覚えておいて、打鍵チャンクの組み方を変えることで対応できないか、打ち終わってから考えてみるようにするといいと思います。

打鍵チャンクの組み方の基本的な方針は、
\begin{enumerate}
 \item 左手から右手、右手から左手に移行するところでチャンクを切らない
 \item 打ちにくいチャンクは分解する
\end{enumerate}
といったところだと思いますが、いくらでも例外はあると思いますし、何度も打ってみて「こういう風に認識すると打ちやすいな」と感じる組み方をしていけば、特に問題は無いと思います。

\subsection{英語のパターンを把握する - 接辞・語根}

\subsubsection*{接辞、語根}

英語には「接辞(接頭辞・接尾辞)」「語根」と呼ばれる構成要素があります。語学に詳しいわけではないので正確な定義や意味は分かりませんが、ともかく、接辞とは単語の前や後ろにくっついて単語の一部を成すもの、語根とは単語の意味の基本となる構成要素のことです。各ひとつずつ例を挙げれば、接辞には mis-(接頭辞。例: mistake)や -tion(接尾辞。例: ignition )、語根には alter-(例: alternate)などがあります。これらの要素は大体そのまま単語に現れるので、ある程度覚えておくことで、素早く文字列を認識し、打鍵チャンク認識を適切に行うための足がかりとなります。-tion という接尾辞があることを知っていて、普段から「-tion は打ちやすいな」と意識していれば、nutrition, ignition, position, acquisition など、様々な単語の中に登場する tion の部分でスムーズに加速できるようになる、ということです。

このように、英語の構成要素を知っておくことは英語タイピングの助けとなります。代表的なものをいくつか紹介しておきます。語根は多岐に渡りすぎていて、ひとつひとつの出現頻度が低いため、今回は接辞のみ、それも個人的に重要だと思うものに限って紹介します。

\subsubsection*{重要な接頭辞}
接頭辞はそれほど重要ではないですが、多少は覚えておくと役に立つと思います。
\begin{itemize}
 \item pre- 前に precise
 \item de- 離れて、下に decide
 \item inter- 間に intermediate
 \item in- ~の中に、上に insist
 \item un- ~でない unable
 \item non- ~でない nonsense
 \item re- 再び resume
 \item in- ~でない inadequate
 \item mis- 誤った mistake
 \item per- 完全に perfect
 \item ex-(e-) 外へ extend
\end{itemize}

\subsubsection*{重要な接尾辞\footnote{\url{http://www.chonmage-eigojuku.com/tangohen/column8.html}}}

接尾辞は非常に重要です。英語タイピングでは絶対に単語の後にスペースが入るので、接尾辞をスムーズに打てればスペースも打ちやすくなり、かなりタイムを稼げます。
\begin{itemize}
 \item -ability, -ibility, -bility 名詞 能力 availability
 \item -able, -ible, -ble 形容詞 能力 capable
 \item -age 名詞 集合、状態、動作 baggage
 \item -al, -ial 形容詞、名詞 性質、~すること proposal
 \item -ate 動詞、形容詞、名詞 ~する、~させる、~のある、~の職 decorate
 \item -ation 名詞 行動、状態、結果 animation
 \item -ative 形容詞 傾向、性質、関係 talkative
 \item -cle, -cule, -ule 名詞 小さな・個々 article
 \item -ence 名詞 silence
 \item -er, -ee, -eer, -yer 名詞 人、物 runner
 \item -form, -iform 形容詞 ~形の cruciform
 \item -free 形容詞 ~のない carefree
 \item -ful 形容詞 満ちている faithful
 \item -fy 動詞 ~化する beautify
 \item -hood 名詞 状態 職業 身分 childhood
 \item -ically 副詞 性質 critically
 \item -ice 名詞 性質 justice
 \item -ician 名詞 職・人 musician
 \item -ing 現在分詞 動名詞 running
 \item -ish 形容詞・動詞 性質・状態・属性 childish
 \item -ive, -itive 形容詞 傾向・性質 native
 \item -ize, -ise 動詞 ~化する criticize
 \item -less 形容詞 ~ない 出来ない careless
 \item -looking 形容詞 ~に見える odd-looking
 \item -ly 副詞 形容詞 finally
 \item -meter 名詞 計測 barometer
 \item -ness 名詞 性質 状態 happiness
 \item -or, -our, -tor 名詞 ~する人 動作 性質 状態 actor
 \item -osity 名詞 性質 jocosity
 \item -ous, -eous 形容詞 ~の多い ~性の courageous
 \item -ple 形容詞 倍・重 simple 単純な
 \item -ship 名詞 状態 身分 能力 関係 membership
 \item -th, -eth 名詞 形容詞 序数を作る 抽象名詞を作る fifth truth
 \item -tion 名詞 動作・状態・結果 ignition
 \item -tious 形容詞 動作・状態・結果 ambitious
 \item -tude, -itude 名詞 性質・状態 aptitude
 \item -ty 名詞 性質・状態・程度 beauty
 \item -ulous 形容詞 ~の傾向のある garrulous
\end{itemize}

\subsubsection*{加速パターン}

これらの接辞の内、個人的に非常に重要だと思っている加速パターンを挙げておきます。英語タイピングをする際に、これらのパターンで加速を逃していないか確認してみてください。数にすればほんの少しですが、これらをマスターするだけでも英語タイピングの能力は上がると思います。

\begin{itemize}
 \item pre-
 \item mis-
 \item -tion, -ation
 \item -ically
 \item -ive, -itive
 \item -ous, -tious
\end{itemize}

他にも自分なりの加速パターンを見つけ、その構成要素がどういった意味を持っているのか、どのような単語に接続するのか、など色々と調べてみると面白い&上達の役に立つかもしれません。

\section{おわりに}

英語タイピングについて、ここまで色々と書かせていただきました。妥当性や一般性に欠ける部分も多かったかと思いますが、英語タイピングの基本はおおむねカバーしたつもりです。2, 3章で挙げた競技(練習サイト)と、5章で挙げたポイントさえ把握していれば、ある程度効率的に練習していくことができるのではないでしょうか。英語タイピングの魅力については伝えきれなかったような気がしますが、ぜひ実際に打ってみて、日本語タイピングとの様々な違いに阻まれながら手探りで上達していく楽しさ、世界中のタイパーと技能を競い合う楽しさを味わってほしいです。

ここまで読んでくださった皆さん、ありがとうございました。この記事をきっかけに少しでも英語タイピングに興味を抱いてもらえたり、既に英語タイピングを始めている方の参考にしてもらえれば幸いです。また、この記事の内容に関して疑問、意見、批判などありましたら、気軽に連絡してください。twitter( @vuttar )、ブログへのコメント、メールなど、連絡手段は何でも構いません。

最後に、宣伝(?)となってしまいますが、この記事とは別に「タイピング練習論」も寄稿させていただいています。この記事以上に曖昧で客観性に欠ける、もはや観念的と言ってもいいような内容になっていますが、私のタイピング観を思い切りぶちまけています。ぜひそちらも読んでいただき、様々な意見や批判をいただければと思います。
