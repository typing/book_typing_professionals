\section{打鍵トレーナー}

\subsubsection*{えむ 氏}
\noindent 打鍵トレーナー、Type Masters の開発者。
打鍵トレーナーは誰でもワードを自作し、自分のホームページ上にタイピングゲームを置くことができたため、ブログ以前のホームページ文化で絶大な人気を誇った。今なお自作ワードを練習する環境として利用されている。

\question{はじめに、えむさん個人とタイピングの馴れ初めについて教えて下さい。}

\answer{えむ}{もともと PC が好きで、よく自作していたんですよ。で、タッチタイピングができたらかっこいいんじゃないかな、ということで練習し始めたのがきっかけです。}

\question{先に PC ありきだったのですね。個人的にタイピングゲームはプレイされていたんでしょうか。}

\answer{えむ}{ソースネクストの特打とかタイプコップ(現・特打コップ)をやっていましたね。タッチタイピングができるようになったのもこれらのソフトのおかげです。}

\question{特打は僕もプレイした記憶があります。では続いて、打鍵トレーナー制作のきっかけについて教えて下さい。}

\answer{えむ}{一番のきっかけは、Flash を使ってタイピングソフトをレンタルサーバーの形式で提供していたところが無くなったから、ですね。かなり昔なのでそのサイトの名前は忘れてしまいましたが。それで、自分の HP に設置できるタイピングソフトが無くなってしまったので、んじゃ自分が作って提供してみようかと思ったわけです。}

\question{打鍵トレーナー以前にそのようなサービスがあったとは知りませんでした。ではブラウザ上で動くものを作るために JavaScript と Perl\footnote{プログラミング言語のひとつ。} CGI でというのは必然の流れだったんですね。}

\answer{えむ}{はい。設置できる環境が一番多いからですね。もともと打鍵トレーナーは「自作 HP に設置してもらう前提」で作成したソフトです。レンタルサーバーのほとんどのサーバーが自作 CGI(Perl)を使用することができますので、それを考えると当然の選択だったかと。}

\question{その後さらに流行することになる「ブラウザゲーム」の体裁をとるタイピングゲームのはしりとなった訳ですが、当時としては「ブラウザ上で」という部分にこだわりは何かあったのでしょうか。}

\answer{えむ}{一番の理由としては、自分で HP を作り始めたところでしたので、その学習のためということがありました。その次の理由としては、どうしてもランキングを作りたかったというのがあります。もうかなり前のことなので知っている方はほとんどおられないかと思いますが、打鍵トレーナーの前身で「TypeMasters」というコンテンツを作って公開していました。これは他人のプレイしたタイピングの速度を事前に記録しておき、そのデータを元にタイプ速度勝負ができるというものでした。個人的に、他の人と競争ができる、という点がすごく好きなんですよ。だからこそ色々な人の記録が残せるブラウザでのソフト開発を行いたかったんです。}

\question{TypeMasters、当時プレイさせて頂いてましたよ。確かにみんなでプレイできるゲームというと今でもブラウザゲームが一番手っ取り早い方法ですね。ではそんな望むタイピングゲームを開発・公開していく上で、印象深い出来事など何かありましたでしょうか。}

\answer{えむ}{特に印象深い出来事はないですが、個人的によくこんなの作ったなとは思います。システムエンジニアへの職種変更をするために前職を退職したところで、ちょうど時間があったからこそできたものだと思いますし、普通に仕事しながら作れるものじゃないな、とも思います。}

\question{ちょうどよい環境が整った時期だったのですね。そして打鍵トレーナーですが、ワードやルールの設定が容易であったため、今でも多くのサイトにコンテンツとして設置されていますね。また打鍵トレーナーを利用し、色々な種目を用意してタイピングゲームサイトとして公開されている所もあります。このような各種打鍵トレーナーを見に行ったりはされていましたか?またそれらを見て感想や、思うことなどありますでしょうか。}

\answer{えむ}{開発した当初は、打鍵トレーナーリンクも自分の HP 上に設置していました。そのため、ちょこちょこと見に行ったりはしていました。設置してもらうのが目的で作ったので、設置してもらえるだけでありがたいですよ。それと、終了時に自動更新しないようにしてサーバー負荷が高くならないように作ったので、その分色々な種類のワードで分けたものを設置しているサイトさんが目立ちましたね。}

\question{これはすごいと思った改造打鍵トレーナーなどありますか。}

\answer{えむ}{ソフトとしてはある程度完成した状態のものだと思いますので、それほどものすごく改造しているものは見たことがないですね。知らないだけかもしれませんが。}

\question{打鍵トレーナーのランキングに自分のサイトの URL が載せられるため、これを利用してサイトを持つユーザ間の横のつながりが促進された面があると思っています。あの世代のホームページ文化(相互リンク・ウェブリングなど)を意識していた面というのはあるのでしょうか。}

\answer{えむ}{これは全く意識していませんでした。毎月 HP リンクを付けに足跡を残してくださる方もおられますがこういう使い方もあるんだな、と逆に教えられた感じです。}

\question{打鍵トレーナーで、うまく作れたと自信がありアピールしたい点、逆にここはもっとうまくやれたのではないかという気になっている点などありますか。}

\answer{えむ}{うまく作れたという思う部分はないですね。HTML と JavaScript の学習の一環として作ってみたものということもありますし、ソースも自慢できるものではないです。もっとうまくやれたのではないかという点は、ワードの編集画面ですね。あれは、一文ごとではなくて、一度に全文を編集できるように作るべきだったと思っています。}

\question{学習の一環、習作にしては完成度の高いソフトだと思います。ランキングに集計期間があり、定期的にリセットされるという要素も打鍵トレーナーのいい部分だと思っているんですが、これは何かアイデアがあっての仕様なんですか?}

\answer{えむ}{これはただ単に、無限にするとデータがあふれてしまうからという理由だけです……。}

\question{当時はサーバの容量制限なども厳しかったですしね。結果オーライという感じで(笑)あとは、初速度平均ランキング機能や打ち切りランキング機能が比較的最近の更新で追加されました。これは何が動機になったのでしょうか。}

\answer{えむ}{一文が終わったら次が表示されて、の繰り返しソフトなのでどうしても長文連続入力の速度とは違う部分がありますよね。だったら、逆手にとって反応速度のランキングを作ってみたら面白いかも、というのが初速度ランキング作成の由来です。打ち切りについては、要望をいただいて対応したような気がします。}

\question{2010 年まで更新が続いていましたが、不具合修正も一段落したのでしょうか、最近は更新がありませんね。もう大きなバグが見つからない限りバージョンアップはないのでしょうか。}

\answer{えむ}{もう作成から 8 年以上も経つソフトですし、大きなバグはないでしょう。バージョンアップという点では、一つ心残りなのが、「っ」で始まるような単語の入力で X からの入力で対応できていない部分がある点です。これは作成当時の PC の性能と JavaScript の動作が遅かったためもあって、入力候補を削って少しでもスピードアップさせるためにした対応なのですが、今となっては後悔だけです。対応するにもかなり大変な修正となるため、手が出せていない状態です。バージョンアップするとすれば、その対応ぐらいでしょうか。後はもう更新するくらいなら新しく別のソフトを作り直すでしょうね。}

\question{やはり当時は色々制約が多かったのですね。今後について言えば、一時期 URL にアクセス不能になり、打鍵トレーナーの配布元がなくなってしまうのではと心配しましたが、復活し、安心しました。これからもしばらくはあの(自宅サーバの)URL で公開を続けて頂けると思ってよいのでしょうか。}

\answer{えむ}{公開はできる限り続けます。しょぼい HP で申し訳ないですが……。}

\question{最後に、何かご自由にメッセージを頂けたらと思います。}

\answer{えむ}{某大学で試験の一部としてタイピングの記録を残すのに使っていただけたり、タイピングランキング大会で使用いただけたり、ということを聞くととてもうれしく、作ってよかったなぁと思う次第です。これだけの古いソフトが今のブラウザでも動くというのも感謝してます (笑) また、今回打鍵トレーナーの内容を取り扱っていただいてありがとうございました。}

\question{こちらこそ、ありがとうございました。}
