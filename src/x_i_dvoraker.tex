\section{Dvoraker - 配列と競技タイピング}

\subsubsection*{Quvota 氏}
\noindent 鍵盤配列や Input Method に造詣の深い知的タイパー。
Dvorak 配列でタイプウェル英単語の頂点に 5 年間もの間君臨し、非標準配列を用いた競技タイピングの象徴的存在となった。他タイプウェル種目でも軒並み Z を出している万能実力者でもある。

\question{英語タイピングのトップであったタイパーとしてのお話、各種配列についてのお話など伺えればと思います。本日はよろしくお願いします。}

\answer{Quvota}{よろしくお願いします。}

\question{まずはいち競技者として、当時のお話を伺いたいと思います。}

\answer{Quvota}{競技というかタイプウェル自体を始めたのは、高校に入ってからなんですよね。パソコンとか好きな人ではあったんで、キーボードはもちろん打ってましたけど。競技を始める前は身の回りに自分より速い人はいなかったので、普通に「自分は速い人間だ」みたいなことを思っていました。}

\question{当時は Qwerty だったわけですよね。}

\answer{Quvota}{そうですね。特に何のひねりもなく、あるがままの配列を打っていて。あの頃はまだ PowerBook、昔の Mac のノートパソコンで、ポッチが\key{K}と\key{D}にあって……とかいうやつを使ってて。中学の頃は部活なんかもやらずにのほほんとしてたんですけど、高校に入って、パ研(パーソナルコンピュータ研究部)という部活に、パソコン嫌いじゃないしということで入ってみたら、そこにタイプウェルをやっていた連中というのがいて。}

\question{その中で既にタイプウェルが流行っていたんですね。}

\answer{Quvota}{かなり昔からタイプウェルをやる人というのがいらっしゃったようで。当時露骨にやっていたのは chokudai とか豆大福とかですね。ただ豆さんよりも前に、結局誰かよくわからなかったんですけど、ランカーがいるらしくて。情報の授業なんかで使うコンピュータースペースという部屋があって、(Windows) NT がガーっと並んでいて、適当なパソコンを使うことができたんですけど、その当時に適当にタイプウェル英単語の基本 1500 が一回だけ打ち切ってあって、見ると ZI とかっていう記録がポコッってあったり(笑)そういうことがあったんで、なんか多分、ランカーが上の方にもいたようなんですけど、誰だかは本当わからないです(笑)}

\question{魔物が棲んでいた(笑)}

\answer{Quvota}{そんな感じで。それで最初タイプウェルやってみたら全然打てないし、\key{Space}押さなきゃいけないしで結構アレだったんです。自分が一番速いと思っていたのになんだこいつらは、という感じでちょっと悔しくて始めてみた、というのがきっかけですかね。}

\question{では一番最初に入った競技はタイプウェルなんですか。}

\answer{Quvota}{そうですね、タイプウェルですね。他にタイピングゲームとかもろくにやったことなかったので、本当にタイプウェルから入ってそればっかりやってたって感じですね。}

\question{どこかの地点で Dvorak 配列に切り替えられるわけですよね。}

\answer{Quvota}{そうですね、その次の日ですね。翌日とかに。}

\question{翌日(笑)}

\answer{Quvota}{タイプウェルをやってみて、タイピングって面白いんだなとちょっと思ってきた時に、その当時に既に Dvorak を使ってたやつがいて。この配列を使うとどうやらもっと速く打てるらしいというような感じの情報を色々と見せてくれて。この配列を使うと、もっと速く打てるのかなと思って始めたんです(笑)当時 Qwerty で全然競技はやってなくて、実質的にタイプウェルをそもそも Dvorak で始めたという感じでした。}

\question{そして数年後、タイプウェル英単語で圧倒的な力を放ち始めるわけですが。}

\answer{Quvota}{あれはその……一年アホみたいに時間ができた年があったので、アホみたいにやっていたというのが実情なんですけど(笑)}

\question{やはり時間はかけないと記録は出せないですよね。}

\answer{Quvota}{僕は体力がない人間なので、在学時代はみんなでわいわいチャットなりをしながら夜更かしをし、眠い状態で朝むりやり学校に行って、授業中も寝て、帰ってクタクタになって……っていう状態、それでやってるのと、いくらでも寝られて、朝起きて「よしっ」って打つのではもう全然違って。まとめて良いコンディションで練習時間を確保できるようになると、伸びるような気がしますね。僕は結構乱打気味だったので……というか、ひたすら乱打でしたね。X とか出る前からトプスピが 1000 超えてたとかそういう(笑)あの頃は結構グリーディに速くなろうとしていました。当時はずっとスピードを上げようとしていて、それに加えて安定性をつけたら、お、伸びた、っていうのが僕の印象ですかね。}

\question{英単語であれだけの記録を出せたのは Dvorak だったから、という部分はどれくらいあると思いますか。もし仮に Qwerty でやっていたとしたら、あの記録は出ただろうか、という訊き方の方が良いかもしれません。}

\answer{Quvota}{どうなんでしょう。多分出ないと思いますけど、ただそれは僕がそう思うだけっていうのはあるのかなぁと。つまり、ある配列を打っていることによって、その配列に体が慣れてくるという面があると思うんです。なので、なかなかその(初めから Dvorak ではなく Qwerty で競技に取り組んでいたらという)想像はできないですね。けど、やっぱり自分に関して言えば Dvorak だからあの記録が出たんだとは思いますね。}

\question{GANGAS に登録されている記録をメモして来たんですけど、種目別のタイムが……。
\begin{itemize}
 \item 基本 24.771 (ZF)
 \item  A-F 26.200 (ZG)
 \item  G-P 26.226 (ZG)
 \item  Q-Z 26.263 (ZG)
\end{itemize}
}

\answer{Quvota}{平らですよね。これはもう平らにしようと努力して平らにしてるんですけど(笑)結構 ZG も出すのが大変で……まあ拡張は圧倒的に基本とはやってる量が違うので、あまり打ち込んではないんですけど。}

\question{いくら Dvorak だからといって、この三種拡張が同程度に打ちやすく、記録がこれほど揃うことはないよなぁと思っていました。}

\answer{Quvota}{単に全部 ZG に載せたいとか総合 ZF にしたいとか、そういう類の根性で揃えただけで、こうなった後はもう全然打ってないですね。基本はあの(最高記録を出した)後も打ったりはしてたんですけど……でもまあ今はテルさんが、とんでもないタイム出して完全に神と化してますし(笑)}

\question{そのあたり、長年公式にはトップであった Quvota さんからみて、どうなんだろうという話も是非聞きたいです。}

\answer{Quvota}{あの領域は見えないですね。彼の基本 22 秒台……あれは意味がわからないですし、ちょっとあれは異次元という感じですね。彼は眠れる獅子というか、もう随分若い時からちゃんとやり込んでいましたもんね。JIS かなで。それからしばらく隠れてましたから……隠れている間に、色々と、なんでしょう……数年サボるとあんなにうまくなるんですかね(笑)}

\question{やはり中学くらいの頃までに一度やり込んだことがある……というのは大きいような印象がありますね。}

\answer{Quvota}{多分そうだと思いますね。あとは、それにかける情熱とかもなかなか。大学生なんてのは結構、情熱を傾けやすい時期だとは思うんですけど、何か集中してやるってなると、身体の可塑性\footnote{変形が、元に戻らないという性質。}とかもありますから。中学とか若い時からいっぱい打ってれば良かったんだろうなーと思ったことはあります。僕なんか、手が自由に動かないので。痩せた時に一緒に筋肉も落ちちゃって。生っちょろい手になってますし。}

\question{ご謙遜を(笑)スキル的な話題として、最適化ですね。Dvorak では最適化を使っていたんでしょうか。}

\answer{Quvota}{してないですね、基本的に。ただ Qwerty では\key{C}\key{E}の\key{C}とか僕左手人差し指を使って取っちゃうんで、そういう意味では厳密に標準運指ではないです。ほぼ標準ではありますけど。}

\question{ローマ字を打つときなど、配列側を弄って\key{C}を\key{K}の代替にするなんてことは……。}

\answer{Quvota}{やってないですね。……あーでも、弄ってるか。これは昔どこかに書いたから、もういいだろうとか思ってやっちゃってるんですけど、ちょっとずるをしていて。基本的に(物理配列は)101\footnote{俗に「英語キーボード」などと呼ばれる、国内では一般的な日本語キーボードよりキー数が少なめのもの。} で打ってたんですけど、\keydouble{Shift}{7}にもシングルクオートを置いていて。Dvorak 本来の位置の\key{'}でもどっちでも打てるようにしてました。タイプウェル英単語を打つときは「I'm」の時だけ\keydouble{Shift}{7}を使って。後のシングルクオートは Dvorak の位置でシフトなしで取る、という、まあちょっとずるいことをしてましたね。}

\question{確かにちょっとメリットがありますね。}

\answer{Quvota}{でも、それ以外はやっていません。\key{K}と\key{C}をひっくり返したりは試してみたんですけど、親にヘボン式でタイピングを教わったので、最初僕はずっとヘボン式で打ってたんですよ。「ちゃ」「ちゅ」「ちょ」を\key{C}で\key{C}\key{H}\key{A}, \key{C}\key{H}\key{U}, \key{C}\key{H}\key{O}(Qwerty では\key{I}\key{J}\key{A}, \key{I}\key{J}\key{F}, \key{I}\key{J}\key{S}という打鍵にあたる)と。Dvorak を使い始めて最初にこれすごい打ちやすいなーと思ったのが強く印象に残っていて、これを崩すのが非常に不愉快だったので、\key{C}と\key{K}を入れ替えるというのはないなと僕は感じていて。結局、やりませんでしたね。}

\question{日本語のローマ字入力で Dvorak 使う人は結構入れ替えていますよね。}

\answer{Quvota}{確かにここで打てるといいとは思います。だから「か」「く」「こ」は\key{C}を使って最近は打ってます。けどもう(\finger{4}の指で)\key{K}を使うのが、完全に自分の中でリズムになってしまっていて。確かにローマ字入力ではそっちの方が速くなるかなと思いながら、いまいち踏み切れてないですね。いや、そういう意味では、その辺の打ち分けをちゃんとやってなくて。頭の固いような意見を言うんじゃないかと思うんですけど、「ん」を\key{X}\key{N}で打ったりとかもしないんですよ。だから最適化とかは実はあんまり……。}

\question{Dvorak という非標準配列を、ほぼ標準運指で使っていたと。}

\answer{Quvota}{あの当時はリアフォ (Realforce) ではなくパンタグラフを使っていたので、パンタグラフ特有の同時押し(スライド打鍵)はやったりしました。縦に並んでる\key{G}\key{H}とかは、うまい叩き方するとワンストロークで入るという。ただ、そうじゃなくも打てるようにはしてましたね。あれは転んだときにミスのリカバリが大変なので、基本的には素直に打っていると。}

\question{僕 Dvorak は X 中盤くらいまでしか打ってないんで、全然感覚が違うとは思うんですけど、ローマ字を打つと「き」がつらいんですが(Qwerty でいえば\key{V}\key{G})。}

\answer{Quvota}{「き」はつらいですね。ってか Dvorak でローマ字入力は本当につらいので。僕はタイプウェルは国語 R だろうと思って、ひたすら国語 R ばっかりやってて。英単語なんかあんまやってなかったんですけど。僕はもう(最初から Dvorak で競技タイピングをしていたので)こういうもんだと思ってずっとそれでやってましたけど、英単語をやってから国語 R に戻ると、打てないですからね(笑)本当につらいと思いますよあれは。}

\question{「Dvorak は右に子音、左に母音がまとまっているから日本語ローマ字入力でも左右交互打鍵で打ちやすい」という話がありますけど、あれは競技タイピングにおいては大嘘ですよね(笑)}

\answer{Quvota}{あれは……あれは嘘だと思いますね(笑)そこはまあ、悪い点なんだろうと思います。}

\question{そういえばパンタグラフだったと仰いましたが、使用していたキーボードを伺ってもよいですか。}

\answer{Quvota}{Thinkpad\footnote{かつて IBM が、今は Lenovo が作っているノートパソコンのブランド。設計思想がマニアうけするだけでなく、キーボードなどインタフェース面でも評価が高い。} を使っていて、最初の Thinkpad 600 は非常によいキーボードだったんですけど、残念ながらタイプウェルは……あのハードでは少々重すぎて。まじめに打ってた時は初速が 0.4 秒切ってたんですよ、デフォルトで。でもあれで打つと、ほとんど 1 秒くらいになる。どういう順番で画面上にオブジェクトが描画されていくかが見ていてわかるという(笑)タイマーもひどいことになって、グラフもすごいガタガタで(笑)……と実用に耐えなかったので、文章を打つのには使ってたんですけど、残念ながらタイピングには向かなかったので、Thinkpad R40 というのを高校に入ってずっと使ってて、実質的にはほとんどそれで競技タイピングをやっていたという印象ですね。}

\question{やや話を戻して、下世話なギャラリー的な話になりますけど、テルさんの登場です。結構センセーショナルだったんですよ、ギャラリー的な立場の僕の中で。Quvota さんの Dvorak だからこそ出せたと思っていた記録を、テルさんが Qwerty でポンと超えて、\footnote{厳密には、非公式にではあるが勃起教教祖氏が 2006-2007 年当時に Quvota さんの記録を抜いている。}もっと行けるもっと行けるって全部超えていくっていうのは。}

\answer{Quvota}{いや、あれは破られて然るべきだと思っていたので。というか未だにやっぱりタイプウェルのランキングは煮詰まってないなぁ、というのは常に思うところです。}

\question{当時としても、抜かれないとは思っていなかった。}

\answer{Quvota}{全然思っていないですね。「全然やってる人いないなぁ」と思っていたので。MADRIGAL さん\footnote{タイプウェルオリジナルで 4 年間もの間、総合トップであった。}の記録とかもそうですけど、記録がバンとあるとそれが人類限界みたいに思って、その中で自分はどれくらいになるかなぁみたいな感じの所、あるじゃないですか。}

\question{確かに、トップの人を基準にして見てしまいますね。}

\answer{Quvota}{そうやって考えてしまっているだけで、もっと行ける余地はあるんだろうなとは僕は思ってましたから、いずれ抜かれるだろうとも思っていて。まあ、あんまりサクッと抜かれてしまったので「わ、わあ」という気はしましたけどね(笑)}

\question{サクッとなんでしょうか(笑)人の事を言うのはアレですが、テルさんは練習量がすごいですよ。努力型のタイパーだと思います。}

\answer{Quvota}{一日どれくらい打ってるんだろう。僕は標準的には一日、数十万打鍵くらいは打ってたんですよ。}

\question{数十万打鍵!(笑)}

\answer{Quvota}{でも僕は、かなりだらだらと練習する人間だったので。しかも自分に体力がないことがわかっていて。基本的に記録って最初しか出ないんですよ。疲労が蓄積していない状態、出だしの何回かで記録が出る。最初に記録が出なかったら、後はもう、惰性でだらだらだらだら打ってるだけという感じで。}

\question{経験値になればいいやという感じですか。}

\answer{Quvota}{あとはだらだら打ってることが楽しかったので。}

\question{クールに記録を出していそうなイメージだったので意外ですけど、当時はジャンキーだったんですね(笑)}

\answer{Quvota}{そうですよ。単にひたすら打ってるだけという馬鹿な生活をしてましたね。モチベーションも……まあやりたい時にやるって感じでしたから。}

\question{本当に意外です(笑)と、競技的なお話はこれくらいにしまして、次は配列の方向で伺いたいと思います。}

\answer{Quvota}{特に話したいことがあるわけじゃないですけど、もっと Dvorak やる人が増えればいいのになーとは思います。Dvorak の話をしようとしたときに、未だに……いや最近 Dvorak について言及しようとする人ならさすがにないかと思うんですけど、「母音子音が左右交互だから打ちやすいんだよ」とかいうのがコモンセンスだと、やっぱり悲しいので、その辺は使っている人が増えると理解が深まって色々出てくるんじゃないかと。そういう意味で、是非皆さんやって下さいって感じですね。}

\question{一方最近はほとんど Qwerty を使ってるとも伺ってますが。}

\answer{Quvota}{僕はもう今 Dvorak 打てないですから(笑)}

\question{そんな(笑)}

\answer{Quvota}{まあ打てないって言っても一時間くらいやればそこそこ打てるんですけど。}

\question{「競技レベルには打てない」という。}

\answer{Quvota}{ですね。僕は、モノの認識力が悪いんだと思うんですけど、かなり集中して見ないと先が読めないので、先まで読んでバッファにためていけて、それがスラスラと手で出てくるようなレベルに維持するのは結構大変で。手が物理的に動くかどうかよりも認識面で、しばらく打っていないと厳しくなってしまいますね。}

\question{配列を切り替えて使うようになるとそれぞれの配列で打鍵を組み立てる能力って落ちますよね。}

\answer{Quvota}{異様な速さを求めなければ、敷居を下げれば……うーん、Machine くらいの速度で良いのであれば、切り替えは割と簡単だと思うんです。だけど国語 R の Z とかそれくらいの速さを維持しながらさくさく切り替えていくってのは……なかなか厳しいかなという。逆にそうでなければ、基本的には問題ないです。一単語ずつとかでも切り替えできる……でしょうね。だからそういうゲームができたら面白いんじゃないかと思うんですけどね。}

\question{\ruby{斑鳩}{いかるが}タイピング\footnote{「斑鳩」という自機のモードを白と黒で切り替えるというシステムを持つ STG がある。配列をそのようにスイッチングしながら打つゲーム、ということ。}ですね(笑)}

\answer{Quvota}{あれは実際あったら面白いと思うんですけど(笑)だから、一配列しかやったことがない人からすると、おそらく「別の配列のをやった後に戻ってこられるのか」という点がすごく心配になると思うんですけど、そんなのは全然心配するような話ではない、というのがまずみんなに浸透してほしいと思います。}

\question{他にはどんな配列を習得されていたんですか。}

\answer{Quvota}{NICOLA や飛鳥……は全然できるようにならなかった。特に親指シフトは普通の JIS かなみたいな押せばいいだけのものより習得に時間がかかる印象がありますね。僕がまじめにやらなかったからなのかもしれないですけど。}

\answer{Quvota}{あと AZIK …… AZIK というか Dvorak ベースの方の ACT ですね。あれもやりかけましたけど、あんまり深入りはできませんでしたね。(認識や打鍵の)切れ目が変わるっていうのは、僕は乗り越えられなくて。僕は音で認識してるんで、思わぬ組み合わせが一打鍵に潰れるっていうのが非常に難しくて。その地点が結局僕は越えられず。打てば(拡張なしの Dvorak より)何ランクか落ちたようなスコアは出るんですけど、それじゃやっている意味はないし……今打たれている配列の中では、省入力という意味で一番先に進んでるとは思うんですが。}

\question{あの方向でやれば、魔改造はできそうですね。}

\answer{Quvota}{とは思うんですよね。僕は越えられなかったんですけど、あの困難を部分的に越えている人もいるというので。だんだん無連想な方向に行くとは思うんです。そもそも Qwerty とかを出発点にする必要がなくなってきて。あの方向でどれだけがんばれるんでしょうね。果たして高速打鍵になったときに頭の中で対応できるのかっていうのが未知数です。僕はかな入力程度の複雑さでもかなり上達するのに時間がかかったので、あれほど動きが複雑になってきて、困難が増えてくると……殊にあれは先読みというか、文字の構造の認識というかが、普通ちょっと先まで見ていればいいのが、ずっと先まで見て、ここがこういう風な打鍵に潰せる、という風に常に認識していく必要があるという。議論としては最適化のコストの議論と同じような話だとは思うんですけど、それをさらに複雑にしたものが必要になる。実際どれくらいが現実的なのかという点は興味のあるところですし、気になります。}

\question{漢直については。}

\answer{Qvuota}{漢直は……うーんどうだろう。キーボードを使い、慣れた配列で打つというのはかなりリソースを使わないで入力できている感覚があるんですが、果たして漢直とかがそういう領域まで行けるのかっていうのが疑問ですね。}

\question{普通のローマ字入力とかかな配列ですと、音までしか認識する必要はないので「しゃべる」感覚ですけど、漢直は文字までイメージしなきゃいけないので「書く」感覚ですよね。}

\answer{Quvota}{そうですね。「書く」まで入るとちょっとどうなのかなと。まあ、やっていないことに対してあれこれ言うのはアレですね。(漢直は)ひらがなを覚えて挫折した人間なので(笑)}

\question{片手配列も色々やってらしたかと思います。}

\answer{Quvota}{片手用 Dvorak とかちょっとやったりはしましたね。チョイ入力も触ったりはしました。}

\question{Twitter で見かけて、結局謎が解けなかったんですけど、「righthandedantisymmetriqwerty」というのは。右手用ということはわかるんですが。}

\answer{Quvota}{片手用 Qwerty で、親指シフト化して、シフトを押すと右手範囲と左手範囲が裏返るっていう配列です。}

\question{あー、シフト面が片手範囲の反対側になっていて、手首を移動する代わりにシフトでパタパタ切り替えられるというような。}

\answer{Quvota}{です。全然ひねりもなく。あれは学習コストがゼロなんで。鏡に映すのって簡単じゃないですか。それでやってみようと思い立って、やってみたってだけなんで、全然画期的なものではないですね。どっちかっていうと右手ではなく左手だけで打てる配列を作りたい……っていうか Qwerty を左手だけで打つ訓練はしたいんです。どうしても\key{Ctrl}を非常によく使う生活をしているので。でもキーボードは残念ながら右の方が領域が大きく、どうしても右上が届かない。なので実用的にどうするべきなのかなぁ、という部分で困っています。今は配列が打てることよりも、それで実生活ができることのほうが自分にとって重要度が高いので。}

\question{色々とすごいお話でした。配列を作るお話が出たので、その方向でもう少し伺いたいです。}

\answer{Quvota}{自分で作ると作って満足して、自分でがんばって考えるから別に頑張って覚えるまでもなく体が覚えていて、よし一仕事終わった、みたいな感じで結局やりませんよね(笑)W/H さんの ZIS かな\footnote{JIS かなのような四段・小指シフトで高速打鍵を目的にした、まだ名前しか決まっていない完全なる妄想配列。}はどうなんでしょう。僕は非常に期待してますけど。}

\question{いやぁ全然でして(笑)実際に JIS かなより速い物ってのはいくらでもできると思うんですけど。}

\answer{Quvota}{どんなに乱暴に作っても(JIS かなよりは)速くなると思うんですよ(笑)速さっていう方向で行って、現実的な今のタイピングゲームの枠組みというか、考え方……をひっくり返さないでもできる配列となると JIS かな系一択だと僕は思うので。}

\question{配列屋さんの手法、各指の使用率の統計を取ったりというような視点についてはどうでしょう。悪いことを言うつもりはないですが、僕たちからするとやや見方が違うわけじゃないですか。}

\answer{Quvota}{まあ、見えている世界が違いますよね。第 0 近似\footnote{本質的でない部分をあえて無視して考えること。}としてはアリなのかなという気はします。けど、それはあくまで第 0 近似であって、いわゆる僕たちみたいな立場の人間からすると、その先にある特徴で色々選びたい。だけど、そこまで議論が深まっていないなぁという見え方をすることがありますね。彼らの言っていることというのも一理あるというか、下手なことをしていない分、彼らはマトモであって、自分で使いたいと思って、それにあったものを頑張って作っているという意味で、非常に良い仕事をしていると思うんです。ただちょっと、僕たちの求めているものではないよね、と思うことはあります。実際僕らのような人達向けではない、と断言している方もいらっしゃいますから、その通りなんだろうなと。}

\question{一方我々の考えるような、ピーキーな配列の作り方となると、現時点で正しいのかどうか確信は持てないにしろ、ある程度正しそうだと言えそうな要素というのは判明してきているわけじゃないですか。認識ですとか、打鍵の構成コストですとか、Z とかになると運動・運指レベルでの経験値ですとか、力学的な限界ですとか。もちろんその頃にはワードと配列がマッチしているか否か、というのも大きな要素になっているし。この辺のタイパー的な経験論と数理的手法をうまく組み合わせながら設計していくと速い配列というのを目指していけるのかなぁとは思っているんですけど。}

\answer{Quvota}{究極的にどういうのがいいのかっていうのは、それこそ弱い指は使わないっていう方針ですごい記録を出している人もいますし、全部使う方針で出してる人もいますし、ちょっと議論が分かれるというか。一つの正解というのはなくて、設計思想で分かれるところなのかなという気がします。}

\question{完璧で一般的なものではなくても、定量評価のモデル\footnote{配列に点数をつける方法のようなもの。}を、タイパー的にまあまあ意味のあるようなモデルを、提示していけたら良いなとは思ってるんですけど。}

\answer{Quvota}{結局、何を求めているのかっていうのが問題ですね。速く打てるのかどうかっていうのは、まあスカラー\footnote{複数の要素を持つベクトルではなく、大きさのみを持つ量。}に落ちるという意味では楽ですけど、ちょっと情報潰しすぎですからね。もうちょっと何かあったらいいなと。でも、そもそもそこがすべての困難ですよね。……多分、タイパーの人達っていうのは細かいことを考えすぎているとは思うんです。もうちょっとざっくりと評価できる軸というのが見えてくるといいなぁと。}

\question{勉強になります。(この本もそうですが)今の現役の一部は、理論派の人はモデル作り、手法まとめ、学習・改良とかで理工的に攻めていて、それとは別にガチガチの競技者の人は自分の感覚で特化配列のような配列を作りましたとか、運指について議論しますとかで模索していったりしています。これがうまく融合すると、……と言うのは簡単で実はとても難しいんですけど、融合できれば、進歩していけそうかなと思って。面白い状況だと思ってます。}

\answer{Quvota}{是非やりたいなぁ……できあがった配列を、みたいな(笑)}

\question{Quvota さん何でもできるんですからお時間あれば手伝って下さいよ(笑)}

\answer{Quvota}{うーん、そういう情報がまとまっている所があるといいなとは思います。今北産業というか。コンテンツを端から消費している立場からすると、追っかけるのがなかなか大変なので。}

\question{学会じゃないですけど、そういう成果発表の場がまとまってあると便利そうですよね。……と、うまくまとまった所で、そろそろ締めに入りたいです。何か伝えたいことなど、ありますでしょうか。}

\answer{Quvota}{さっきも言ったことになるんですが、みんな Dvorak 打とうよと。いや Dvorak じゃなくてもいいんです。マジになって別な配列打っている人っていうのがいたらいいなと思いますね。はっきり言って二番目の配列なんて、すぐうまくなるじゃないですか。だから簡単に第二第三の配列ができるようにはなる、と思うんですけど……でも多くの人が中途半端なところでやめてしまうというか。かたや Qwerty とかでは Z 出したいと言っている割に、他だとまあ Machine 出たし満足かなとか。Machine すら行かなくて多くの人は Genius とかだと思うんですけど。第一の配列では、Genius で Genius(天才) だとは思いませんし Machine で Machine(機械) だとは思いませんよね。そういう感覚で、もっとこの配列やらなきゃっていう感覚で、やりこんでくれる人がいっぱいいればなぁと。}

\question{耳が痛いです(笑)}

\answer{Quvota}{いや、僕は高校に入って、自分がある程度自信を持っていたいずれに関しても、ヤバいやつがいるっていう時に、これは誰しも人生のどこかしらのタイミングでみんな色々あると思うんですが、アイデンティティの危機に瀕するわけです。そうした時に、僕の素直な解決策が「誰もやってないのをやる」だったんですよ。誰もやってないのをやれば第一人者になれますよね(笑)後に人が続くかどうかってのは完全に別問題ですけど。そういう風な感じで、横道にそれるというか、変なことをやる人がもっといると、いいんじゃないかなぁと思うんです。中途半端にやった感想とか聞いても、別に面白くないと思うんですよ。極めた人の話って、聞いてると面白いじゃないですか。別にタイピングに限らずに。みんなと同じじゃなくて、みんなと違った経験を持っている人っていうのがもっと増えて、もっと好き勝手なことを言ってくれたらいいのになと思います。}

\question{皆さん競技者なので、同じフィールドで戦いたいという気持ちがあったりするのでしょうね。}

\answer{Quvota}{後ろめたさというのはやっぱり若干あって。確かに気持ちが通じる人が近くにいるというのは割と重要です。一人でやるのはなかなか。実は僕は、振り返ると、周りに Dvorak をやっていた人が他に三人いたので。非常に恵まれていて、ぬるい生活だったんです(笑)だからまあ、そういう機運も高まるといいですね。}

\question{ありがとうございました。}
